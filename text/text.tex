\documentclass{article}
\usepackage[a4paper]{geometry}
\usepackage{amssymb}
\usepackage{euscript}
\usepackage{graphicx}
\graphicspath{{images/}}
\usepackage{color}
\usepackage{epsfig}
\pagestyle{empty}

\begin{document}
\title{Aerogel experiment in F2}

\author{}

\date{\today}
\maketitle

\subsection*{Measurements}

There is a flopper wire with aerogel sample which moving in constant magnetic field.
Position of the wire is measured by applying additional RF gradient field and measuring
response from a RF coil attached to the wire.

There are two types of measurements:

\begin{itemize}

\item  Response on the RF frequency is recorded
contineously as a function of temperature while wire is driven by a
low-frequency sine excitation with constant frequency. Lock-in phase is
adjusted to have a single-component response.

\item  A second low-frequency lock-in is detecting the signal on the
flopper excitation frequency. Response is measured as a function of
flopper excitation frequency, amplitude and temperature.

\end{itemize}

\subsection*{Processing of the first type of measurements}


Driving force is proportional to the excitation signal:
$$
F = K\,U_e^0 \cos(\omega t)
$$

Flopper wire coordinate is described by some equation of motion:
$$
m \ddot x = -k x - g \dot x + F
$$
probably with some additional non-linear terms.

We are measuring a value proportional to $x(t)$ (maybe with
some small high-order terms).
$$
R(x) = Ax + \ldots
$$

There is also a static measurement ($x=\mbox{const}$),
$$
R(x) = F/k  =  K\,U_e^0 / k
$$


\end{document}
