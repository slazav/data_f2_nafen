\documentclass{article}
\usepackage[a4paper]{geometry}
\usepackage{amssymb}
\usepackage{euscript}
\usepackage{graphicx}
\graphicspath{{images/}}
\usepackage{color}
\usepackage{epsfig}
\pagestyle{empty}

\begin{document}
\title{Aerogel experiment in F2}

\author{}

\date{\today}
\maketitle

\subsection*{Measurements}

There is a flopper wire with aerogel sample which moving in constant magnetic field.
Position of the wire is measured by applying additional RF gradient field and measuring
response from a RF coil attached to the wire.

There are two types of measurements:

\begin{itemize}

\item  Response on the RF frequency is recorded
contineously as a function of temperature while wire is driven by a
low-frequency sine excitation with constant frequency. Lock-in phase is
adjusted to have a single-component response.

\item  A second low-frequency lock-in is detecting the signal on the
flopper excitation frequency. Response is measured as a function of
flopper excitation frequency, amplitude and temperature.

\end{itemize}

\subsection*{Processing of the first type of measurements}

Scripts for processing the first type of measurements are located in {\tt
flopper\_dacfiles} folder. The main program is {\tt
process\_flopper\_dacfile}. It gets a data folder and date of the first
measurement (it is not recorded in files). Then it reads and all {\tt
*.dat} files one by one:

\begin{itemize}
\item Time, drive, and signal comlumns are extracted.

\item Signal is splitted by N parts (N is set by command-line option, default is 1).
This is done to increase time resolution and reduce effect of possible instability
of the drive frequency (to be checked - is it important or not).

\item Drive frequency is determined by doing FFT and fitting it near the peak
(similarly to how I do it in processing of pulsed fork signals, see in the code)

\item Drive and signal are averaged using N time bins over a single period (N is
set by command-line option, default is 50).

\item The averaged data (time, drive, and signal) are recorded into
<time>.txt files.

\item N first harmonics of the drive and signal are calculated (N is set
by command-line option, default is 3). Signal harmonics with frequency
and time are printed to {\tt stdout}.

\end{itemize}

There is a {\tt test subfolder} where a model signal is created, and processed.




\end{document}
